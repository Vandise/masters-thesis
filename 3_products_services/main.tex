{\let\cleardoublepage\relax \chapter{Products and Services}}

\section{Description of Products and Services}

Cereus aims to become the leader in cutting-edge privacy compliance auditing tools. These tools will assist our customers to quickly and efficiently identify privacy and compliance issues on their websites. Most of Cereus’s core product will be offered as software-as-a-service with additional professional services also being made available.

\subsection{Compliance Auditing}

The Cereus compliance auditing system is a series of rules and processes configured by the user to ensure their websites meet the compliance standards that they have defined. It can be configured to scan websites during the businesses development cycle, automatically through the API services, or manually through the Cereus user interface. There are five main components to the compliance auditing system: the confirmation system, crawler, rules engine, report, and recommendation engine.

\subsubsection{Confirmation System}

The Cereus confirmation system, Figure \ref{3_products_services/figures/confirmation_system}, prevents organizations from conducting audits on domains that they do not own or manage. This will ensure Cereus's customers cannot use the auditor to evaluate their competitors websites and privacy practices. When a new property (website) is added through the Cereus user interface, the user will be guided through a series of processes to confirm ownership of the domain before allowing an audit to be conducted. This process is dependent on the plan the customer is subscribed to. Any attempt to scan an unconfirmed domain will be rejected.

The free tier is a highly restricted plan that limits the capabilities of the auditor. This tier is intended for less technical business owners running a small website. For initial confirmation of ownership of the domain, free tier users will be required to own an email address associated with the domain they are requesting to audit. In many instances, an email such as "webadmin@example.com" are dedicated to the management of the domain. Once the customer creates the property in the Cereus user interface, they can then request a validation email be sent to their inbox with a confirmation link. When confirmed, the customer will be provided an HTML metadata tag to be included on the pages they would like to be scanned.

In the event the property on the free tier expires or is sold, Cereus will no longer be able to audit the website due to the metadata tags not being present on the site. 

\figuremacro{3_products_services/figures/confirmation_system}{Property confirmation system}{The Cereus website confirmation system to ensure the organization owns the domain prior to scanning.}

All additional tiers offer unrestricted access to Cereus's services, which can provide insights into the privacy operations of the company. To confirm ownership of the property, a TXT DNS record defined by Cereus's systems will be provided to the client. In the event a domain is acquired by a new party, through the sale or expiration of the domain, Cereus will lock access to previous reports and disable auditing services when the TXT DNS record is no longer present. The initial client who set up the domain in Cereus will be notified of the change and can re-validate if the DNS record was removed by mistake.

\subsubsection{Crawler}

The Cereus crawler is a highly intelligent and configurable service that uses artificial intelligence to scan, monitor and classify information from validated websites. This includes all network traffic, cookies set on the page, call-trace information, performance data, request type, response status codes, headers, and redirect flags. This data is then processed by the rules validator and aggregated by the report generator. Cereus then makes recommendations on actions that should or must be taken to bring the website back into compliance. Unlike competitors, this includes the exact line and position of code where the non-compliance occurred.

\subsubsection*{Configuration}

The crawler can be configured in a variety of ways to meeting an organization’s auditing and compliance needs. From a set or random schedule, organizations can ensure their publicly accessible website are scanned on regular basis to ensure compliance and to verify the websites are only updated through approved channels. Once the data has been processed by the rules validator and the report has been generated, the customer will receive an email notification with the audit report. Audits may also be conducted manually by a user through the Cereus user interface or triggered through the API services.

Cereus's customers may also specify a geographic location for which the crawler will originate from. Depending on the geographic location of the visitor, privacy regulations can differ and the company may apply a different set of business rules. Running the crawler in a targeted geographic location allows Cereus's customers to validate their website's behaviors.

\subsubsection*{Data Extraction and Transformation}

The network and cookie data are extracted in a semi-structured format that can be translated to a flat SQL tables for processing by the report generator. Data received by the proxy server can be consolidated with debug information sent by the browser. Figure \ref{3_products_services/figures/data_transform} outlines the data collection and transformation process.

The crawler captures the requested URL for users to construct rules on the domains, locations, and query parameter. The requested URL is broken down into the requested protocol, domain, path, query, and location hash. This fragmentation will significantly reduce the amount of processing required by the rules and recommendation engines. It also allows customers to optimize their reports by establishing aggregation or exclusion rules based on part of the URL. In the instance data is sent to the server, for example, in a POST request: the data is converted to a HashMap and stored as JSON in the metadata column.

All request and response header information, depending on the browser meta event, is formatted as JSON and stored in the headers column. Signals such as the Do Not Track header signal (DNT) can be sent from the browser to indicate that the user would prefer privacy rather than personalized content \cite{mdn.2020}. Site partners may respond to the DNT header, or some other setting, that the customer can monitor with the Cereus rules engine.

\figuremacrorotate{3_products_services/figures/data_transform}{Crawler data transformation}{The extraction and transformation of data received by the Cereus crawler.}{1}

\subsubsection{Rules Engine}

Organizations have the option to establish rules associated with a network request or cookie to determine whether or not they meet compliance standards. These rules can be configured to target specific geographic location to determine if an entire URL, domain, protocol, query path, or parameter, based on the specified condition, meets compliance expectations (Figure \ref{3_products_services/figures/rules_definition}).

Multiple conditions may be applied to a single URL or cookie as an OR conditional. Rules may also be grouped to establish an AND conditional between two or more rules. Cereus's rules engine supports matches for values that: contain a value, equal a value, does not contain a value, does not equal a value, or whether or not the value matches a regular expression. Based on specified conditions, the user can specify whether or not to flag the request as compliant or not.

\figuremacro{3_products_services/figures/rules_definition}{Rules definition interface}{The Cereus rules engine can be configured to determine whether or not a request meets compliance standards based on the conditions specified by the user.}

These rules, by default, are set at an organizational scope. All properties under the organization, when the rules engine processes a crawl, will have the same rules applied. Rules may also be overrode at a property level when exclusions are needed.

\subsubsection{Reporting}

Once processed by our proprietary Artificial Intelligence and patent-pending algorithms the Cereus audit reporting provides insights into the requests made on the site. This incorporates the response status code, the type, size of the information exchanged, the amount of time for the request to complete, and whether or not the request met compliance expectations (Figure \ref{3_products_services/figures/sample_report}). Any information that has no data or rules associated with it appears as a question mark (?) icon.

Each row can be broken down to dive into the information associated with the request. The deep dive includes the headers associated with the request, a cache of the original rules associated with the request (and their validation status), query parameters, data sent to the server, cookies, and the initialization chain. Web administrators and compliance managers will be able to quickly reference the audit report and identify where the compliance infringement originated.

\figuremacrorotate{3_products_services/figures/sample_report}{Cereus audit report}{The Cereus audit report provides clean, familiar, interface for privacy compliance teams.}{1}

The report filters can be used to dynamically query information from it. Users can check for specific URLs, whether or not requests were compliant, the associated category with a URL, or the page in which the information was found. Customers will receive a notification when the audit identifies requests out of compliance and can use the filter functionality to quickly pull up the request and rules information. 

\subsubsection{Recommendation Engine}

Cereus can make suggestions for requests and cookies that have yet to be classified by the organization. This classification system is powered by the categorizations of requests and cookies by other organizations. The system will also crawl the domains, paths, and cookie hosts to grab meta information to improve the accuracy of the recommendations. A risk score will also be assigned to the requests and cookies based on whether or not other organizations have flagged it as necessary to their operations.

\subsubsection*{Risk Score}

When a new request or cookie is identified on a website, through an initial or later scan, an associated risk score with allowing it to load in a list of geographic locations. This score is merely a suggestion based on the operations of other organizations and no action needs to be taken.

\[ s_{risk} = \frac{r_n}{R} \]

The risk score is computed as the number of records classified as necessary \( r_n \) divided by the total number of records \( R \). This will always result in a ratio between 0 and 1 in which intervals of \(\frac{1}{3}\) will determine if the request will be rated as: low, intermediate, or high risk.

To reduce the possibility of organizations incorrectly flagging a request due to the scoring system, the risk score will only be provided when a sample size of at least 20 organizations have classified the request or cookie.

\subsubsection*{Categorization Recommendation}

Cereus will provide categorization recommendations for requests and cookies based on meta information extracted from the request or cookie's origin. The recommendation engine will also incorporate organization classification information pertaining to the request or cookie. Classification within Cereus's internal systems are expected to be single words or small phrases, much like meta tag information present on a web page. This information can best be represented as a bag of words. There's no context to meta tag data or the collection of classifications entered by users, so the representation of language or order has no meaning \cite{manning.2008}.

The recommendation engine is powered by a Bernoulli document model. The model takes a document and partitions it into a feature vector of binary elements. If a word is found in the document, it will receive a value of 1, otherwise 0. It does not take into account the frequency of a word, but whether or not the word is present. The model can then calculate the probability of a word occurring in a document with a specific classification, as well as taking into account the probability of it not occurring.

To save reduce the computational power required to provide recommendations, the model calculates estimated probabilities for a request or cookie belonging to a category.

\(\hat{P}(w_{t}|C_{k})\) defines the estimated probability that the word \(w_{i}\) occurs in a document, \(D\), with the classification (\(C\)) \(k\). The estimated probability that the word \(w_{i}\) not occurring is \(1 - \hat{P}(w_{i}|C_{k})\). \(V\) represents the model's vocabulary and \(v\) consists of the feature vector of the document. The product of the probability of each item (\(i\)) in the feature vector occurring or not occurring determines the overall estimated probability of the document being classified as class \(k\) (\(\hat{P}(v|C_{k})\)).

\[
  \hat{P}(D|C_{k}) = \prod_{i=1}^{V}[v_{i}\hat{P}(w_{i}|C_{k}) + (1 - v_{i})(1 - \hat{P}(w_{i}|C_{k}))]
\]

There are two parameters for this model: the probabilities of each word in the document class (\(\hat{P}(w_{i}|C_{k})\)) and its prior probabilities \(P(C_{k})\). The estimated probability that a word \(w_{i}\) occurs in a document is the number of documents \(n\) classified as \(k\) divided by the total number of documents \(N\) classified as \(k\).

\[
  \hat{P}(w_{i}|C_{k}) = \frac{n_{k}(w_{i})}{N_{k}}
\]

Where the prior probability of class \(k\) can be estimated as the relative frequency of documents containing class \(k\).

\[
  \hat{P}(C_{k}) = \frac{N_{k}}{N}
\]

The output of the model will be an array of estimated probabilities. Each probability in the array is associated with a category defined within Cereus's system. The category with the highest probability is recommended to the user.

\subsection{Notifications and Alarms}

Users can configure Cereus to notify the specified stakeholders in the event that an audit discovers new requests or cookies, compliance checks fail, or when a new scan is scheduled. To reduce the likelihood of users ignoring these notifications, they can be configured on an organizational level or per property. These settings ensure that notifications are sent only to the relative parties and reduce email clutter.

\subsection{API Services}

Professional and Enterprise customers may use Cereus's application programming interface (API) services to automate their privacy operations. Customers who actively curate new content, use a continuous integration (CI) system to control website deployments, or may not follow a strict deployment schedule can use the API to integrate Cereus into their daily business processes.

\subsubsection*{Crawl Requests}

The crawl request endpoint allows customers to trigger a crawl against a property. By default, this will scan all paths defined on the property. Users also have the option to specify the path(s) they would like to crawl in the event they want to target only new or updated pages.

\subsubsection*{Web Hooks}

When an audit has completed, it will notify stakeholders stating whether or not compliance checks were successful. Cereus can also notify other computer systems. Through web hooks, customers are able to configure Cereus to send audit status messages to their computer systems. The customer can then process the message and take action against the messages.

\subsection{Professional Services}

Cereus's professional services will serve a clientele requiring guidance on privacy regulations applicable to their operations. Cereus will consult the client to identify the best plan for the client and assist with the configuration of Cereus's tools to reflect the client's needs. Training services for Cereus's products will also be provided.

\section{Competitive Advantages and Disadvantages}

The privacy industry, though relatively new, already consists of some large organizations servicing a significant amount of the market share. In April of 2020, OneTrust had a 35.5\% stake in the data privacy management software market \cite{onetrust.2020}. Cereus intends to enter the privacy software market by automating an otherwise manual process: compliance auditing.

\subsection{Competitive Advantages}

\begin{enumerate}
  \item \textbf{Founders are experienced software engineers}
  
  Cereus's founders are experienced software engineers. Providing software-as-a-service requires the software to be developed, maintained, and readily available. Cereus's founders will be able to construct its systems to be scalable and maintainable to accompany organizations of any size.
  
  \item \textbf{Founders are experienced product managers}

  Cereus's founders are experienced product managers. The founders are able to clearly define Cereus's product vision and prioritize feature requests to meet customer requirements.

  \item \textbf{Founders have a background in data science}

  Cereus's founders have a background in data science. Cereus will collect, process, and store large amounts of information for their customers. When large datasets come into play, storage costs, data integrity and performance issues are common. With a formal background in data science, Cereus's founders are able to mitigate all of these problems.

  \item \textbf{Cereus provides actor stack traces}

  In the event a request or cookie fails to meet company compliance standards, Cereus provides stack trace information that allows the user to locate the file and exact line of code that triggered the validation failure. This saves the company time and money from having to manually trace the source.

  \item \textbf{Cereus extends cookie compliance}

  In addition the industry standard of cookie compliance (user consent is required only for cookie tracking), Cereus also provides insights into network requests. This provides a full-scope audit of the website and how information is shared between it and its partners.

\end{enumerate}

\subsection{Competitive Disadvantages}

\begin{enumerate}

  \item \textbf{Cereus is not a consent management platform}

  Cereus does not provide consent management services for its customers; it augments existing consent management platforms and solutions. Cereus's customers will need to purchase a consent management solution or implement their own.

  \item \textbf{Adoption requires legal support}

  Tools used to automate legal compliance often require privacy attorneys to validate whether or not the tools are effective \cite{bloomlaw.2019}. Cereus will require legal support in order for companies to adopt its services.

\end{enumerate}

\section{Pricing Structure}

Cereus's primary target market is medium to large-sized organizations managing multiple websites. The company, however, does offer a flexible pricing structure, including a free tier, to accommodate companies of any size. Table \ref{table.cereus.pricing} provides an overview of each tier and services provided.

\subsection{Free}

Cereus offers auditing for websites, once a month and for a single page, for free. Free-tier users will be authorized to register one website within Cereus but will not be able to benefit from the company's API, web hook, notification, or professional services.

\subsection{Standard}

Todo.

\subsection{Professional}

Todo.

\subsection{Enterprise}

Todo.

\begin{sidewaystable}
\centering
\begin{tabularx}{\textheight}{|c|c|c|c|c|c|c|c|c|c|c|c|}
  Tier & Subscription & Price & Users & Websites & Pages & Scans/Mo & Prof.Services & API & Web hooks & Notifi. & Rec. \\

  \hline

  Free & Monthly & \$0.00 & 1 & 1 & 1 & 1 & \xmark & \xmark & \xmark & \xmark & \xmark \\

  \hline

  Personal & Monthly & \$39.00 & 1 & 1 & 5 & 1 & \xmark & \xmark & \xmark & \cmark & \xmark \\

  \hline

  Standard & Monthly & \$249.00 & 5 & 5 & 5 & 2 & Additional & \xmark & \cmark & \cmark & \xmark \\

  \hline
  
  Professional & Monthly & \$2,200.00 & 25 & 25 & 15 & 5 & \cmark & \cmark & \cmark & \cmark & \cmark \\

  \hline
  
  Select & Yearly & Custom & N & N & N & N & Opt. & Opt. & Opt. & Opt. & Opt. \\

\end{tabularx}
\caption{Cereus pricing tier and services provided.}
\label{table.cereus.pricing}
\end{sidewaystable}

%test

\section{Industry Background}

Lorem ipsum dolor sit amet, consectetur adipiscing elit, sed do eiusmod tempor incididunt ut labore et dolore magna aliqua. Ut enim ad minim veniam, quis nostrud exercitation ullamco laboris nisi ut aliquip ex ea commodo consequat. Duis aute irure dolor in reprehenderit in voluptate velit esse cillum dolore eu fugiat nulla pariatur. Excepteur sint occaecat cupidatat non proident, sunt in culpa qui officia deserunt mollit anim id est laborum

\section{Target Market Segment}

Lorem ipsum dolor sit amet, consectetur adipiscing elit, sed do eiusmod tempor incididunt ut labore et dolore magna aliqua. Ut enim ad minim veniam, quis nostrud exercitation ullamco laboris nisi ut aliquip ex ea commodo consequat. Duis aute irure dolor in reprehenderit in voluptate velit esse cillum dolore eu fugiat nulla pariatur. Excepteur sint occaecat cupidatat non proident, sunt in culpa qui officia deserunt mollit anim id est laborum


