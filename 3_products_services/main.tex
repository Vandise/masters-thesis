{\let\cleardoublepage\relax \chapter{Products and Services}}

\section{Description of Products and Services}

Cereus aims to become the leader in cutting-edge privacy compliance auditing tools. These tools will assist our customers to quickly and efficiently identify privacy and compliance issues on their websites. Most of Cereus's solutions will be offered as software-as-a-service, though professional services will also be available.

\subsection{Professional Services}

Cereus's professional services will serve a less technical clientele or those requiring guidance on privacy regulations applicable to their operations. Cereus will consult configuration the client to identify the best plan for the client and assist with the configuration of Cereus's tools to reflect the client's needs. Training services for Cereus's products will also be provided.

\subsection{Compliance Auditing}

The Cereus compliance auditing system is a series of processes configured by the user to ensure their websites meet the compliance standards that they have defined. It can be configured to scan per the businesses development cycle, automatically through the API services, or manually through the Cereus user interface. There are five main components to the compliance auditing system: the confirmation system, crawler, rules engine, report, and recommendation engine.

\subsubsection{Confirmation System}

The Cereus confirmation system, Figure \ref{3_products_services/figures/confirmation_system}, prevents organizations from conducting audits on domains that they do not own or manage. This will ensure Cereus's customers cannot use the auditor to evaluate their competitors websites and privacy practices. When a new property (website) is added through the Cereus user interface, the user will be guided through a series of processes to confirm ownership of the domain before conducting an audit is authorized. This process is dependent on the plan the customer is subscribed to. Any attempt to scan an unconfirmed domain will be rejected.

The free tier is a highly restricted plan that limits the capabilities of the auditor. This tier is intended for less technical website administrators running a small website through a content management system (CMS). For initial confirmation of ownership of the domain, free tier users will be required to own an email address associated with the domain they are requesting to audit. In many instances, an email such as "webadmin@example.com" are dedicated to the management of the domain. Once the customer creates the property in the Cereus user interface, they can then request a validation email be sent to their inbox with a confirmation link. When confirmed, the customer will be provided an HTML metadata tag to be included on the pages they would like to be scanned.

In the event the property on the free tier expires or is sold, Cereus will no longer be able to audit the website due to the metadata tags not being present on the site. 

\figuremacro{3_products_services/figures/confirmation_system}{Property confirmation system}{The Cereus website confirmation system to ensure the organization owns the domain prior to scanning.}

All additional tiers offer unrestricted access to Cereus's services, which can provide insights into the privacy operations of the company. To confirm ownership of the property, a TXT DNS record defined by Cereus's systems will be provided to the client. In the event a domain is acquired by a new party, through the sale or expiration of the domain, Cereus will lock access to previous reports and disable auditing services when the TXT DNS record is no longer present. The initial client who set up the domain in Cereus will be notified of the change and can re-validate if the DNS record was removed by mistake.

\subsubsection{Crawler}

The Cereus crawler is a highly configurable network and cookie monitor that extracts information from websites. This includes all network traffic, cookies set on the page, call-trace information, performance data, request type, response status codes, headers, and redirect flags. This data is then processed by the rules validator and aggregated by the report generator. Cereus can then make recommendations on actions to take based on the audit.

\subsubsection*{Configuration}

The crawler can be configured by the client to automatically scan a website based on the company's software release schedule. Once the data has been processed by the rules validator and the report has been generated, the customer will receive an email notification with the audit report. Audits may also be conducted manually through the Cereus user interface or triggered through the API services.

Cereus's customers may also specify a geographic location for which the crawler will originate from. Depending on the geographic location of the visitor, privacy regulations can differ and the company may apply a different set of business rules. Running the crawler in a targeted geographic location allows Cereus's customers to validate their website's behaviors.

\subsubsection*{Data Extraction and Transformation}

The network and cookie data are extracted in a semi-structured format that can be translated to a flat SQL tables for processing by the report generator. Data received by the proxy server can be consolidated with debug information sent by the browser. Figure \ref{3_products_services/figures/data_transform} outlines the data collection and transformation process.

The crawler captures the requested URL for users to construct rules on the domains, locations, and query parameter. The requested URL is broken down into the requested protocol, domain, path, query, and location hash. This fragmentation will significantly reduce the amount of processing required by the rules and recommendation engines. It also allows customers to optimize their reports by establishing aggregation or exclusion rules based on part of the URL. In the instance data is sent to the server, for example, in a POST request: the data is converted to a HashMap and stored as JSON in the metadata column.

All request and response header information, depending on the browser meta event, is formatted as JSON and stored in the headers column. Signals such as the Do Not Track header signal (DNT) can be sent from the browser to indicate that the user would prefer privacy rather than personalized content \cite{mdn.2020}. Site partners may respond to the DNT header, or some other setting, that the customer can monitor with the Cereus rules engine.

\figuremacrorotate{3_products_services/figures/data_transform}{Crawler data transformation}{The extraction and transformation of data received by the Cereus crawler.}{1}

\subsubsection{Rules Engine}

Organizations have the option to establish rules associated with a network request or cookie to determine whether or not they meet compliance standards. These rules can be configured to target specific geographic location to determine if an entire URL, domain, protocol, query path, or parameter, based on the specified condition, meets compliance expectations (Figure \ref{3_products_services/figures/rules_definition}).

Multiple conditions may be applied to a single URL or cookie as an OR conditional. Rules may also be grouped to establish an AND conditional between two or more rules. Cereus's rules engine supports matches for values that: contain a value, equal a value, does not contain a value, does not equal a value, or whether or not the value matches a regular expression. Based on specified conditions, the user can specify whether or not to flag the request as compliant or not.

\figuremacro{3_products_services/figures/rules_definition}{Rules definition interface}{The Cereus rules engine can be configured to determine whether or not a request meets compliance standards based on the conditions specified by the user.}

These rules, by default, are set at an organizational scope. All properties under the organization, when the rules engine processes a crawl, will have the same rules applied. Rules may also be overrode at a property level when exclusions are needed.

\subsubsection{Reporting}

The Cereus audit report formats the network traffic and associated cookies in a clean, tabular, format. This initial overview provides insights into the requests made on the site, response status code, the type, size of the information exchanged, the amount of time for the request to complete, and whether or not the request met compliance expectations (Figure \ref{3_products_services/figures/sample_report}). Any information that has no data or rules associated with it appears as a question mark (?) icon.

Each row can be broken down to dive into the information associated with the request. The deep dive includes the headers associated with the request, a cache of the original rules associated with the request (and their validation status), query parameters, data sent to the server, cookies, and the initialization chain. Web administrators and compliance managers will be able to quickly reference the audit report and identify where the compliance infringement originated.

\figuremacrorotate{3_products_services/figures/sample_report}{Cereus audit report}{The Cereus audit report provides clean, familiar, interface for privacy compliance teams.}{1}

The report filters can be used to dynamically query information from it. Users can check for specific URLs, whether or not requests were compliant, the associated category with a URL, or the page in which the information was found. Customers will receive a notification when the audit identifies requests out of compliance and can use the filter functionality to quickly pull up the request and rules information. 

\subsubsection{Recommendation Engine}

Cereus can make suggestions for requests and cookies that have yet to be classified by the organization. This classification system is powered by the categorizations of requests and cookies by other organizations. The system will also crawl the domains, paths, and cookie hosts to grab meta information to improve the accuracy of the recommendations. A risk score will also be assigned to the requests and cookies based on whether or not other organizations have flagged it as necessary to their operations.

\subsubsection*{Risk Score}

When a new request or cookie is identified on a website, through an initial or later scan, an associated risk score with allowing it to load in a list of geographic locations. This score is merely a suggestion based on the operations of other organizations and no action needs to be taken.

\[ s_{risk} = \frac{r_n}{R} \]

The risk score is computed as the number of records classified as necessary \( r_n \) divided by the total number of records \( R \). This will always result in a ratio between 0 and 1 in which intervals of \(\frac{1}{3}\) will determine if the request will be rated as: low, intermediate, or high risk.

To reduce the possibility of organizations incorrectly flagging a request due to the scoring system, the risk score will only be provided when a sample size of at least 20 organizations have classified the request or cookie.

\subsubsection*{Categorization Recommendation}

Cereus will provide categorization recommendations for requests and cookies based on meta information extracted from the request or cookie's origin. The recommendation engine will also incorporate organization classification information pertaining to the request or cookie. Classification within Cereus's internal systems are expected to be single words or small phrases, much like meta tag information present on a web page. This information can best be represented as a bag of words. There's no context to meta tag data or the collection of classifications entered by users, so the representation of language or order has no meaning \cite{manning.2008}.

The recommendation engine is powered by a Bernoulli document model. This model takes a document and partitions it into a feature vector of binary elements. If a word is found in the document, it will receive a value of 1, otherwise 0. This document model does not take into account the frequency of a word, but whether or not the word is present. This allows us to calculate the probability of a word occurring in a document with a specific classification, as well as taking into account the probability of it not occurring.

We'll let \(\hat{P}(w_{t}|C_{k})\) define the estimated probability that the word \(w_{t}\) occurs in a document, \(D\), with the classification (\(C\)) \(k\). The estimated probability that the word \(w_{t}\) not occurring is \(1 - \hat{P}(w_{t}|C_{k})\). \(V\) will represent our models vocabulary and \(v\) consist of the feature vector of our document. The product of the probability of each item (\(i\)) in our feature vector occurring or not occurring will determine the overall estimated probability of our document being classified as class \(k\) (\(\hat{P}(v|C_{k})\)).

\[
  \hat{P}(D|C_{k}) = \prod_{i=1}^{V}[v_{i}\hat{P}(w_{t}|C_{k}) + (1 - v_{i})(1 - \hat{P}(w_{t}|C_{k}))]
\]

\subsection{Notifications and Alarms}

Lorem ipsum dolor sit amet, consectetur adipiscing elit, sed do eiusmod tempor incididunt ut labore et dolore magna aliqua. Ut enim ad minim veniam, quis nostrud exercitation ullamco laboris nisi ut aliquip ex ea commodo consequat. Duis aute irure dolor in reprehenderit in voluptate velit esse cillum dolore eu fugiat nulla pariatur. Excepteur sint occaecat cupidatat non proident, sunt in culpa qui officia deserunt mollit anim id est laborum



