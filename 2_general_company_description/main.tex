{\let\cleardoublepage\relax \chapter{General Company Description}}

Cereus is a software-as-a-service company offering privacy compliance auditing tools and solutions for businesses that collect customer information through their websites.

{\let\clearpage\relax \section{Unique Value Proposition}}

Consent involves more than just cookies on your website. It involves all information you share with your vendors and partners including network traffic. Cereus actively scans, monitors, and analyzes your website to identify compliance infringements in accordance with your business rules.

{\let\clearpage\relax \section{Mission Statement}}

To remove privacy barriers between companies and their partners, and enable them to communicate in an efficient, compliant way.

%To become the leader in cutting-edge privacy compliance auditing tools. To help customers quickly and efficiently identify privacy and compliance issues on their websites.

\section{Vision Statement}

Proactively detect all privacy compliance infringements before they reach the customer. 

\section{Values Statement}

\begin{itemize}
  \item Reliability
  \item Transparency
  \item Growth
\end{itemize}

\section{Company Goals and Objectives}

The primary goal of Cereus is to establish itself as a thriving company that leads the privacy industry by providing automated, insightful, audits to ensure that companies and their partners are sharing customer information in accordance to their business rules.

\section{Business Philosophy}

Some companies spend millions to establish a privacy program. Everyone else gets Cereus.

\section{Industry Overview}

The privacy technology industry is a rapidly growing field. In 2020, it is the fastest growing technology sector that includes the fastest growing company in the U.S \cite{hughes.iapp.2020}. As more consumers become impacted by massive data breaches in which sensitive, personally identifying information (PII), is exposed; consumer awareness and the call for data processing regulations are expected to be on the rise.

There are already governmental regulations impacting businesses in the U.S. The Health Insurance Portability and Accountability Act (HIPAA) highly regulates patient information and how it is stored. The California Privacy Protection Act (CCPA) regulates the selling of user data collected by a business for consumers in the state of California. The Children's Online Privacy Protection Act (COPPA) imposes requirements on website operators on collecting information from children under the age of 13 years old. Lastly, the General Data Protection Regulation (GDPR) gives European Union citizens the right to manage their information any business has collected on them and requires explicit consent before information is collected.

With all these regulations, foreign and domestic, companies that collect information from their customers are relying on the assistance of privacy technologies to operate within the bounds of new regulations and meet consumer privacy expectations \cite{meehan.forbes.2019}. This has attracted massive funding, and, in July of 2019, OneTrust raised \$200 million in a Series A investment, TrustArc raised \$70 million Series D, Privitar raised \$40 million series B, and BigID raised \$30 million Series B \cite{wood.fpf.2019}.

In four years since its founding, as of August, 2020, OneTrust is valued at \$2.7 billion \cite{hughes.iapp.2020}.

\section{Market Segment Overview}

Any business that operates a website and collects data and analytics on their customers is subject to the regulations in which the customer originates. This also applies to the jurisdiction in which the business operates. Cereus can provide auditing for companies with a single website, to large enterprises with hundreds. It may prove difficult for small businesses with a single website to justify the expense of privacy audits when they are not often the subject of privacy lawsuits \cite{lanou.2020}. Cereus's primary focus will be medium to large organizations maintaining multiple websites. 


\section{Legal Form of Ownership}

Cereus will be organized as a small business corporation (S corporation). As an S corporation, Cereus will be allowed to collect funding and pass off any business profits and expenses to its shareholders without the additional taxes applied to C corporations.

\section{Guiding Principles}

The "Living Principles for Design" framework \cite{brink.aiga.2020} was applied to outline how Cerues can maintain a sustainable design while achieving the company's objectives along the following dimensions:

\subsection{Environment}

The direct environmental impact of Cereus is expected to be minimal. Cereus will provide software-as-a-service (SAAS) and will rely on cloud services to manage its operations. Cloud services, such as Amazon Web Services (AWS), are composed of large computer networks in which infrastructure is shared with other AWS customers \cite{aws.2020}. Cereus's physical hardware is limited to the machines required to manage services running on the cloud platforms.

With Cereus's services operating in the cloud, a central office space for employees is not required and will further reduce the company's environmental impact.

\subsection{People}

The societal impact of Cereus and its services are restricted to the transparency of the companies that use it. Cereus offers detailed reports from its audits that can provide insights into how customer information is shared between a website and its partners. If companies choose to share these reports, their customers will better understand how their information is used in exchange for the services the company provides. This has the potential to improve the relationship between a company and their customers -- potentially even making them more apt to sharing personally identifying information.

\subsection{Economy}

Todo.

\subsection{Culture}

Todo.
