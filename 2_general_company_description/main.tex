{\let\cleardoublepage\relax \chapter*{General Company Description}}
\addcontentsline{toc}{chapter}{General Company Description}

Cereus is a software-as-a-service company offering privacy compliance auditing tools and solutions for businesses that collect customer information through their websites. We augment privacy consent and risk management practices by auditing  websites per the business rules our customers define within our systems. We'll automatically conduct audits based on their release schedule or on-demand.

\hfill

\section{Unique Value Proposition}

We believe that consent involves more than just cookies on our customer's websites. It involves all information they share with their vendors and partners including network traffic. We will actively scan, monitor, and analyze our customer's websites to identify compliance infringements per their business rules.

\section{Mission Statement}

It's our mission to remove privacy barriers between companies and their partners and enable them to communicate in an efficient, compliant way.

%To become the leader in cutting-edge privacy compliance auditing tools. To help customers quickly and efficiently identify privacy and compliance issues on their websites.

\section{Vision Statement}

We will proactively detect all privacy compliance infringements defined by our clients before they reach their customers.

\section{Values Statement}

\begin{itemize}
  \item Reliability
  \item Transparency
  \item Growth
\end{itemize}

\section{Company Goals and Objectives}

Our primary goal is to establish ourselves as a thriving company that leads the privacy industry by providing automated, insightful, audits to ensure that companies and their partners are sharing customer information per their business rules.

\section{Business Philosophy}

Some companies spend millions to establish privacy compliance and risk management program. Everyone else gets Cereus.

\section{Industry Overview}

The privacy technology industry is a rapidly growing field. In 2020, it is the fastest-growing technology sector that includes the fastest-growing company in the U.S \cite{hughes.iapp.2020}. As more consumers become impacted by massive data breaches in which sensitive, personally-identifying information (PII), is exposed; consumer awareness and the call for data processing regulations are expected to be on the rise.

There are already governmental regulations impacting businesses in the U.S. The Health Insurance Portability and Accountability Act (HIPAA) highly regulates patient information and how it is stored \cite{cdc.2018}. The California Privacy Protection Act (CCPA) regulates the selling of user data collected by a business for consumers in the state of California \cite{calleg.2018}. The Children's Online Privacy Protection Act (COPPA) imposes requirements on website operators on collecting information from children under the age of 13 years old \cite{ftc.1998}. Lastly, the General Data Protection Regulation (GDPR) gives European Union citizens the right to manage their information any business has collected on them and requires explicit consent before information is collected \cite{eucomm.2016}.

With all these regulations, foreign and domestic, companies that collect information from their customers are relying on the assistance of privacy technologies to operate within the bounds of new regulations and meet consumer privacy expectations \cite{meehan.forbes.2019}. This has attracted massive funding, and, in July of 2019, the fastest-growing company in the U.S., OneTrust, raised \$200 million in a Series A investment. Other privacy software companies, including TrustArc, raised \$70 million Series D, Privitar raised \$40 million series B, and BigID raised \$30 million Series B \cite{wood.fpf.2019}.

In four years since its founding, as of August 2020, OneTrust is valued at \$2.7 billion \cite{hughes.iapp.2020}.

\section{Market Segment Overview}

Any business that operates a website and collects data and analytics on their customers is subject to the regulations in the area/jurisdiction in which the customer originates. This also applies to the jurisdiction in which the business operates. Cereus provides auditing for companies with a single website, to large enterprises with hundreds. It may prove difficult for small businesses with a single website to justify the expense of privacy audits when they are not often the subject of privacy lawsuits \cite{lanou.2020}. Our primary focus will be medium to large organizations maintaining multiple websites. 

\section{Legal Form of Ownership} \label{legal.ownership}

Cereus will be organized as a small business corporation (S corporation). As an S corporation, we will be allowed to collect funding and pass off any business profits and expenses to its shareholders without the additional taxes applied to C corporations.

\section{Guiding Principles} \label{guiding.principles}

The "Living Principles for Design" framework \cite{brink.aiga.2020} was applied to outline how Cereus can maintain a sustainable design while achieving our objectives along the following dimensions:

\subsection{Environment}

The direct environmental impact of our company is expected to be minimal. We will provide software-as-a-service (SAAS) and will rely on cloud services to manage our operations. Cloud services, such as Amazon Web Services (AWS), are composed of large computer networks in which infrastructure is shared with other AWS customers \cite{aws.2020}. Our physical hardware is limited to the machines required to manage services running on the cloud platforms.

With our services operating in the cloud, centralized office space for employees is not required and will further reduce the company's environmental impact.

\subsection{People}

The societal impact of our company and its services are restricted to the transparency of the companies that use it. We offer detailed reports from our audits that can provide insights into how customer information is shared between a website and its partners. If companies choose to share these reports, their customers will better understand how their information is used in exchange for the services the company provides. This has the potential to improve the relationship between a company and its customers -- possibly making them more apt to share personally-identifying information.

\subsection{Economy}

Cereus's operations are expected to reduce the amount of time required to conduct a compliance audit against websites. These actions will minimize the manual auditing cost and the likelihood of our customers being subject to privacy lawsuits. Our customers can then focus and dedicate more resources towards achieving their goals and growing their business. Our overall economic impact is limited to the actions of our customers and is expected to be minimal.

\subsection{Culture}

We have the potential to influence organizations to be more transparent about the sharing of information on their customers with their partners. Traditionally, data processing and sharing are often confidential and kept internal; but with privacy becoming a concern for consumers -- transparency will soon be an expectation \cite{meehan.forbes.2019}.


