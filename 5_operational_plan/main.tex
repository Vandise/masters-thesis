{\let\cleardoublepage\relax \chapter*{Operational Plan}}
\addcontentsline{toc}{chapter}{Operational Plan}

\section{Implementation Framework}

As discussed in the \nameref{guiding.principles} section, Cereus will follow the "Living Principles for Design" AIGA framework \cite{brink.aiga.2020} to guide implementation efforts. It emphasizes that many of our operations along the interpersonal and cultural dimensions are limited to the transparency of our customers. Though we encourage our customers to share the information in our audits, we ultimately leave the decision to them.

Many of the criteria of AIGA along the environmental dimension don't often apply to a company offering software-as-a-service as we have no physical products. To minimize any environmental impact required by data storage, web hosting, networking, and data processing, we will rely on cloud services for our operations.

In the case of the economic dimension, Cereus aims to automate privacy compliance auditing for websites. This technological advancement will reduce the amount of time and resources required to conduct a privacy audit. We believe that this will allow our customers to focus and dedicate more resources towards achieving their goals and growing their business. Our overall economic impact is limited to the actions of our customers and is expected to be minimal.

\section{Production}

Cereus's founders will develop all software in-house using open-source tools, libraries, and frameworks. Strict quality control measures will be taken, including continuous integration, unit testing, integration testing, and code reviews. All code will be under source control and versioning.

One of the three founders will be in charge of all product development and management. The product manager will receive feedback from our customers and allocate work to the developers based. For each software development iteration (sprint), Cereus will deliver a minimum viable product for each feature designated by the product manager. Customer feedback will further shape how our product operates.

\section{Location}

Cereus will not lease or physically owned any real estate. We offer software-as-a-service through a cloud provider. The founders will be able to contribute to the development of our products and services without any special equipment. 

Physical correspondence to Cereus will operate through a PO Box. 

\section{Legal Environment} \label{legal.environment}

There are no regulations, zoning requirements, or permits required for Cereus to begin operations. Our crawler and auditor collects publicly available, non-personal, information that is not regulated by any legislation. For processing transactions, we partner with Authorize.net, which is audited yearly for PCI DSS compliance \cite{authnet.2020}. Customers under the General Data Protection Regulation request us, or to our credit processing partner to request that their information to retrieved or anonymized.

For system authorization, Cereus requires general information such as the user's email address, a password, and the user's name. We utilize modern security practices by encrypting our databases and hashing salted passwords -- making it extremely difficult to exploit user information from our systems \cite{aprn.security.2015}.

All customers will be required to accept our general terms of use and a release of liability. Cereus, as an auditing company, can only provide insights and suggestions based on the configurations our customers have entered into our systems. We do not provide legal advice nor can we guarantee that our audits ensure our customer's websites meet regulations their websites are subjected to.

\section{Personnel}

For the first year, Cereus will consist of the three founders. The founders will not take a salary or dividends during the first year to allow the company to grow. This will limit initial funding costs and allow the founders to invest profits back into the company. Tasks will be allocated to the founders based on their expertise: data science, software engineering, product management, and marketing.

As discussed in the \nameref{section.admin.costs} section, Cereus will allocate \$200 per month towards general accounting and customer account management. This will be handled by an outside firm.

\section{Inventory}

Cereus will not provide any physical products for distribution. Most products and services will be available immediately after purchase.

\section{Suppliers}

We partner with Amazon Web Services for all our infrastructure and data storage requirements. As a cloud provider, Amazon allows us to rapidly scale our systems based on customer needs with minimal costs and maintenance. Their Service Level Agreement requires that they make commercially reasonable efforts to achieve a monthly uptime percentage of at least 99.9\% \cite{amazon.2019}. If Amazon falls short of this uptime percentage, we will be eligible to receive a service credit. As discussed in the \nameref{section.operational.costs} section, Amazon's cost explorer allows us to conduct a monthly break-even analysis and forecast costs. With predictable billing from our cloud provider, we will be able to make adjustments to our budget and allocate funds as needed.

Cereus also partners with Authorize.net. Though generally more expensive than other credit processing companies, Authorize.net has a consistent 2.9\% transaction fee plus \$0.30 \cite{yowana.2019}. The benefit of Authorize.net, in contrast to other credit card processors, is that it offers protection against chargebacks for \$25.00 per month -- an anticipated issue with customers who subscribe before evaluating our services. Should an issuing bank rule in favor of our company, any associated chargeback fees will be refunded.

\section{Credit Policies}

In the \nameref{transaction.process} section, we mentioned that Cereus will not offer credit to customers. Billing will be conducted every month. Customers will be required to enter billing information for all of our subscription tiers and allow the transaction to successfully process before gaining access to our services. Accounts becoming delinquent will result in our auditing services being disabled for the customer; this will mitigate Cereus incurring any additional costs associated with the account. Accounts delinquent for more than 30 days will result in the customer being locked out from our systems.
