{\let\cleardoublepage\relax \chapter{Marketing Plan}}

% guidelines: https://www.thebalancesmb.com/writing-the-business-plan-section-5-2947030

%========================================================

\section{Industry Background}

\subsection{Market Size} \label{market.size}

According to QY Research, North America and Europe are currently holding the largest share for the privacy management software market \cite{qy.2020}. In 2019, this market size was estimated at a US \$808.8 million and estimated to reach a US \$6.1 billion by the end of 2026 with a compound annual growth rate of 33.1\% from 2021 through 2026. These estimates were based on privacy management software trends in North America, Europe, China, Southeast Asia, India, and Central and South America.

\subsection{Market Trends} \label{marketing.trends}

With the passing of the General Data Protection Regulation (GDPR) for European Union citizens and large data breaches exposing the personal information of hundreds of millions of consumers, privacy awareness has led to the passing of privacy regulations across an increasing amount of countries \cite{privacypolicies.2019}. Other countries, including Brazil, Canada, Australia, and various U.S. states have enacted, or began enforcing, their privacy laws with heftier fines or sanctions. The primary difference between regulations being passed now, in contrast to the past, are the heavy fines and sanctions regulators can impose on companies who are non-compliant \cite{tr.2020}. These sanctions and fines are designed to force businesses to comply with regulations and how they process their customers information.

In a survey conducted by Auxier, Rainie, Anderson, Perrin, Kumar, and Tuner, 62\% of Americans believe it is not possible to go through daily life without having their data collected, 81\% believe that they have very little to no control over their information companies collect about them, and 72\% feel that all, or almost all of what they do online is being tracked \cite{pewresearch.2019}. The survey also revealed that most Americans are not confident that companies would publicly admit to misusing their customers data, but still say data collection and usage is acceptable for processing in some ways.

Citizens of countries belonging to the European Union, 1 year since the passing of GDPR, are rapidly becoming aware of their privacy rights \cite{eucomm.2019}. A survey conducted by the European Union Commission against 27,000 Europeans revealed that 73\% of citizens have heard of at least one of their privacy rights with the highest awareness being the right to access their own data (65\%), the right to correct their information a company has on them (61\% ), and the right to opt-out of direct marketing (59\%).

Other trends involve the usage of smart mobile devices. According to the FTC, information from smart phones, such as location information, is considered sensitive \cite{ftc.mobile.2012}. The location information can be used to provide, unwanted, targeted marketing to consumers based on their movements. Mobile applications can use the hardware identifiers associated with devices to uniquely identify consumers or potentially access utilities no required for the intended purpose of the application \cite{tama.mobile.2012}.

\subsection{Growth Potential and Opportunity}

In section \ref{market.size}, QY Research identified a compound annual growth rate of 33.1\% from 2021 through 2026. Section \ref{marketing.trends} identified emerging trends such as: privacy awareness, information control, and concerns with smart mobile devices. With a rapidly growing market and rising awareness and concerns with privacy, we can anticipate that governments will further regulate how companies can process consumer information. With further regulations, the demand for consent management vendors will grow (as predicted by QY Research) and  automated privacy auditing services to ensure privacy risk is mitigated by the vendors will be in similar demand. 

\subsection{Market Barriers}

Even with governmental regulations, hefty fines, and sanctions; adoption of consent management platforms (CMP) isn't widespread and integrations often don't meet compliance regulations \cite{nouwens.2020}. In an audit conduced by Aarhus, Cambridge, and UCL Universities, of the top 10,000 websites in the UK, only 11.8\% met the minimal requirements set by their audit based on European law. Of this sample, only 20.35\% of websites report to use a CMP. In addition to the top 10,000 websites, in a sample of 1,000 consent management platform vendors, 95.8\% provided either no consent choice or confirmation only.

%\noindent
A long-term study of the impact of GDPR on cookie placement also revealed that \cite{trevisan.2019}:

\begin{enumerate}
  \item 49\% of websites placed cookies before receiving consent.
  \item 28\% of websites didn't provide any consent mechanism.
  \item The percentage of websites violating GDPR stayed constant over the course of 4 years, indicating any consent mechanisms implemented were ineffective.
\end{enumerate}

%\indent

\noindent
The primary barrier for Cereus to successfully provide privacy auditing services will be industry acceptance. With many vendors offering consent management solutions, a majority of integrations failed to meet compliance standards \cite{nouwens.2020}. We will have to differentiate ourselves from consent management providers looking for a piece of a booming market and emphasize that we are not a consent management solution. Cereus is an auditing and risk mitigation solution that evaluates integrations with CMPs. Should a CMP integration fail to adhere to the defined rules within our auditing system, those failures will be reflected in the audit report.

\subsection{Market Changes}

As an automated privacy auditing company, Cereus is moderately affected by fluctuations in the privacy software market due to changes in privacy legislation and regulation. Response to changes will, generally, be the responsibility of our customers -- they will be able to alter their business rules within our systems to ensure that any new legislation is accounted for. To support our customers, our professional services staff will receive training on any new or changes to legislation.

In the event legislation expands or is implemented in geographic locations that we do not yet support, our Engineering staff will make the appropriate changes prior to the enforcement of the legislation.

%========================================================

\section{Products and Services}

Customers with a privacy program may see auditing as the next logical step towards ensuring compliance and privacy risk mitigation measures are being followed across all of their properties. Depending on size of their organization, manual auditing may be a possibility. Customers will also have to consider the costs associated with manual auditing, the possibility for human error, and the frequency in which the audits must be conducted. 

Other customers looking for privacy management solutions may not consider auditing and risk management as the first priority. They understand that their websites need to comply with relevant laws and a consent management platform can assist them with compliance. These customers will look through our products and services and will quickly identify that we do not offer a consent management solution, so why would they choose our services?

We, Cereus, agree that a consent management platform seems like the first logical step when viewing privacy legislation trends and actions your competitors may be taking. We also believe that a consent management platform may not be required given your audience, business practices, and tooling. After conducting an audit and identifying any privacy risks your businesses are susceptible to, you will be able to determine an appropriate course of action. That course may include implementing a consent management platform or possibly using open-source solutions that will mitigate risk.

With the highly customizable products and flexible pricing (even free) we offer, customers will see the potential to: identify whether or not they need a consent management platform, the associated risk of not implementing one, and the effectiveness of their integration should they implement a consent management platform.

%--------------------------------------------------------

\subsection{Auditor and Reporting}

Any customer, whether they have implemented a consent management platform or not, will first look towards how auditing and reporting will be provided. More importantly, what information the audit contains and how it can be distributed.

\subsubsection*{Unique Selling Proposition}

Define your compliance rules and actively monitor all your properties to ensure that they meet your compliance standards.

\subsubsection*{Features and Benefits}

\begin{itemize}

\item Customizable

  Define your compliance rules within our system for our proprietary crawler to validate against your websites.

\item Interactive

  Verify the results of your audit in our interactive report. Identify which of your properties failed to meet your compliance rules and where the infringement occurred on the website -- down to the line of code causing the infringement.

\item Exportable

  Export your audit into a distributable format to share with your stakeholders.

\item Automated

  Schedule audits to reflect your development cycle and quickly identify compliance infringements before they are distributed to your customers.

\end{itemize}

%--------------------------------------------------------

\subsection{Notifications and Alarms}

Customers may wonder how long an audit will take to run against their websites and, unfortunately, there is no definitive answer. Audits depend on how many pages the crawler must scan and the execution speed of your website. All of our customers, except those using the free tier, may specify who, if anyone, will receive notifications of when audits complete for each property, as well as an alarm should compliance checks fail.

\subsubsection*{Unique Selling Proposition}

Allow Cereus to grant you peace of mind by notifying you should one of your websites fall out of compliance.

\subsubsection*{Features and Benefits}

\begin{itemize}

\item Customizable

  Specify which stakeholders, at an organization or property scope, to notify when audits complete, and compliance checks fail for any of your websites.

\item Prompt

  No need to step away from your task to check on the status of your audits, allow Cereus to notify you as soon as your audit is complete.

\end{itemize}

%--------------------------------------------------------

\subsection{Classification and Recommendation Engine}

All our customers, excluding those on the free tier, will see our audit system provide a risk score, residual risk score, and suggest a category for partners running on your website that you have yet to review. Many will ask why such functionality is provided as their legal department will determine the rules associated with a partner. To assist with the evaluation of partners, we've aggregated resources so you don't have to.

\subsubsection*{Unique Selling Proposition}

Allow us to assist you with establishing your business rules by providing insights into the operations of your partners and suggest a course of action for your websites.

\subsubsection*{Features and Benefits}

\begin{itemize}

\item Automated

Cereus will automatically calculate the risk score and residual risk score should a partner load on your site in specific geographic locations. This score is based on the actions taken by our other customers, vendor information, and your privacy policy.

\item Productivity

Make informed decisions about mitigating any risk associated with your partners through the resources we've collected pertaining to their operations.

\end{itemize}

%--------------------------------------------------------

\subsection{API}

Organizations actively developing their websites will look for functionality that enables them to integrate privacy compliance and risk management into their daily operations. Our application programming interface (API) provides a variety of features to support a continuous integration system. This is available for our professional, enterprise, and select customers only.

\subsubsection*{Unique Selling Proposition}

Automate privacy compliance and risk management by integrating our API into your business operations.

\subsubsection*{Features and Benefits}

\begin{itemize}

\item Efficiency

Include privacy compliance and risk management in your continuous integration system. Run audits on-demand and let us send the results to your systems.

\end{itemize}

%--------------------------------------------------------

\subsection{Professional Services}

Not all organizations have the time or resources to dedicate towards running a privacy audit and risk management division. Professional and enterprise customers have the option to allow our professional services staff configure our systems to meet your privacy compliance and risk management requirements.

\subsubsection*{Unique Selling Proposition}

Allow us to assist your organization by configuring our systems to meet your privacy compliance and risk management requirements.

\subsubsection*{Features and Benefits}

\begin{itemize}

\item Support

Staff readily available to assist with your integrations, setup, and general questions.

\item Active Management

Our professional services staff can configure and manage our systems to meet your privacy compliance and risk management requirements.

\end{itemize}


%========================================================

\section{Customers}

Though we offer flexible subscriptions for any sized organization that's looking for privacy compliance auditing. It may, however, be difficult to market smaller companies that our services are worth mitigating the risk of being fined. Our customers will primarily be medium-to-large organizations, with multiple websites, who are currently, or will be, working with a consent management system.

\subsection{Demographic Considerations}

The geographic location of our customers and their client-base are some factors to consider when defining the demographics of our customers. The location of our customers and their client-base define the regulations they have to adhere to. Should an organization chooses to restrict their client-base due to data and privacy regulations, our services may not be required.

%========================================================

\section{Proposed Location}

Cereus will be incorporated in the United States. There will be no leased or physically owned real estate. Correspondence pertaining to Cereus will operate through a PO Box. 

%========================================================

\section{Pricing and Positioning Strategy}

TODO Chapter 2

\subsection{Operational Costs}

https://calculator.aws

aws a1.medium spot instance: 0.0049/hr ( scan servers )
aws a1.medium reserved instance: 0.0255 ( api servers )
aws s3 standard ( up to 50tb), 0.023/gb
aws cloudfront ( first 10tb ), 0.085/gb
aws NAT gateway: 0.045/hr + 0.045/gb
aws RDS on demand ( multi-az db.t3.small ): 0.072/hr
aws elasticache (t3.small): (0.034/hr) 
aws loadbalancer (api): 0.025/hr + 0.008/gb
aws route53: 0.50 + 0.40/mil queries (up to 1 billion)

\subsection{Competitive Analysis}

TODO Chapter 2

\subsection{Return on Investment}

TODO Chapter 2

%========================================================

\section{Sales and Distribution}

TODO Chapter 2

\subsection{Sales Strategy}

TODO Chapter 2

\subsection{Distribution Methods}

TODO Chapter 2

\subsection{Transaction Process}

TODO Chapter 2

%========================================================

\section{Advertising and Promotion}

TODO Chapter 2

\subsection{Advertising}

TODO Chapter 2

\subsection{Sales Promotion}

TODO Chapter 2

\subsection{Publicity}

TODO Chapter 2

\subsection{Web}

TODO Chapter 2
