{\let\cleardoublepage\relax \chapter{Marketing Plan}}

% guidelines: https://www.thebalancesmb.com/writing-the-business-plan-section-5-2947030

%========================================================

\section{Industry Background}

\subsection{Market Size} \label{market.size}

According to QY Research, North America and Europe are currently holding the largest share for the privacy management software market \cite{qy.2020}. In 2019, this market size was estimated at a US \$808.8 million and estimated to reach a US \$6.1 billion by the end of 2026 with a compound annual growth rate of 33.1\% from 2021 through 2026. These estimates were based on privacy management software trends in North America, Europe, China, Southeast Asia, India, and Central and South America.

\subsection{Market Trends} \label{marketing.trends}

With the passing of the General Data Protection Regulation (GDPR) for European Union citizens and large data breaches exposing the personal information of hundreds of millions of consumers, privacy awareness has led to the passing of privacy regulations across an increasing amount of countries \cite{privacypolicies.2019}. Other countries, including Brazil, Canada, Australia, and various U.S. states have enacted, or began enforcing, their privacy laws with heftier fines or sanctions. The primary difference between regulations being passed now, in contrast to the past, are the heavy fines and sanctions regulators can impose on companies who are non-compliant \cite{tr.2020}. These sanctions and fines are designed to force businesses to comply with regulations and how they process their customers information.

In a survey conducted by Auxier, Rainie, Anderson, Perrin, Kumar, and Tuner, 62\% of Americans believe it is not possible to go through daily life without having their data collected, 81\% believe that they have very little to no control over their information companies collect about them, and 72\% feel that all, or almost all of what they do online is being tracked \cite{pewresearch.2019}. The survey also revealed that most Americans are not confident that companies would publicly admit to misusing their customers data, but still say data collection and usage is acceptable for processing in some ways.

Citizens of countries belonging to the European Union, 1 year since the passing of GDPR, are rapidly becoming aware of their privacy rights \cite{eucomm.2019}. A survey conducted by the European Union Commission against 27,000 Europeans revealed that 73\% of citizens have heard of at least one of their privacy rights with the highest awareness being the right to access their own data (65\%), the right to correct their information a company has on them (61\% ), and the right to opt-out of direct marketing (59\%).

Other trends involve the usage of smart mobile devices. According to the FTC, information from smart phones, such as location information, is considered sensitive \cite{ftc.mobile.2012}. The location information can be used to provide, unwanted, targeted marketing to consumers based on their movements. Mobile applications can use the hardware identifiers associated with devices to uniquely identify consumers or potentially access utilities no required for the intended purpose of the application \cite{tama.mobile.2012}.

\subsection{Growth Potential and Opportunity}

In section \ref{market.size}, QY Research identified a compound annual growth rate of 33.1\% from 2021 through 2026. Section \ref{marketing.trends} identified emerging trends such as: privacy awareness, information control, and concerns with smart mobile devices. With a rapidly growing market and rising awareness and concerns with privacy, we can anticipate that governments will further regulate how companies can process consumer information. With further regulations, the demand for consent management vendors will grow (as predicted by QY Research) and  automated privacy auditing services to ensure privacy risk is mitigated by the vendors will be in similar demand. 

\subsection{Market Barriers}

Even with governmental regulations, hefty fines, and sanctions; adoption of consent management platforms (CMP) isn't widespread and integrations often don't meet compliance regulations \cite{nouwens.2020}. In an audit conduced by Aarhus, Cambridge, and UCL Universities, of the top 10,000 websites in the UK, only 11.8\% met the minimal requirements set by their audit based on European law. Of this sample, only 20.35\% of websites report to use a CMP. In addition to the top 10,000 websites, in a sample of 1,000 consent management platform vendors, 95.8\% provided either no consent choice or confirmation only.

%\noindent
A long-term study of the impact of GDPR on cookie placement also revealed that \cite{trevisan.2019}:

\begin{enumerate}
  \item 49\% of websites placed cookies before receiving consent.
  \item 28\% of websites didn't provide any consent mechanism.
  \item The percentage of websites violating GDPR stayed constant over the course of 4 years, indicating any consent mechanisms implemented were ineffective.
\end{enumerate}

%\indent

\noindent
The primary barrier for Cereus to successfully provide privacy auditing services will be industry acceptance. With many vendors offering consent management solutions, a majority of integrations failed to meet compliance standards \cite{nouwens.2020}. We will have to differentiate ourselves from consent management providers looking for a piece of a booming market and emphasize that we are not a consent management solution. Cereus is an auditing and risk mitigation solution that evaluates integrations with CMPs. Should a CMP integration fail to adhere to the defined rules within our auditing system, those failures will be reflected in the audit report.

\subsection{Market Changes}

As an automated privacy auditing company, Cereus is moderately affected by fluctuations in the privacy software market due to changes in privacy legislation and regulation. Response to changes will, generally, be the responsibility of our customers -- they will be able to alter their business rules within our systems to ensure that any new legislation is accounted for. To support our customers, our professional services staff will receive training on any new or changes to legislation.

In the event legislation expands or is implemented in geographic locations that we do not yet support, our Engineering staff will make the appropriate changes prior to the enforcement of the legislation.

%========================================================

\section{Products and Services}

TODO Chapter 2

%--------------------------------------------------------

\subsection{Auditor and Reporting}
TODO Chapter 2

\subsubsection*{Unique Selling Proposition}
TODO Chapter 2

\subsubsection*{Features and Benefits}
TODO Chapter 2

%--------------------------------------------------------

\subsection{Notifications and Alarms}
TODO Chapter 2

\subsubsection*{Unique Selling Proposition}
TODO Chapter 2

\subsubsection*{Features and Benefits}
TODO Chapter 2

%--------------------------------------------------------

\subsection{Rules Engine}
TODO Chapter 2

\subsubsection*{Unique Selling Proposition}

TODO Chapter 2

\subsubsection*{Features and Benefits}

%--------------------------------------------------------

\subsection{Classification and Recommendation Engine}

TODO Chapter 2

\subsubsection*{Unique Selling Proposition}

TODO Chapter 2

\subsubsection*{Features and Benefits}

TODO Chapter 2

%--------------------------------------------------------

\subsection{API}

TODO Chapter 2

\subsubsection*{Unique Selling Proposition}

TODO Chapter 2

\subsubsection*{Features and Benefits}

TODO Chapter 2

%--------------------------------------------------------

\subsection{Professional Services}

TODO Chapter 2

\subsubsection*{Unique Selling Proposition}

TODO Chapter 2

\subsubsection*{Features and Benefits}

TODO Chapter 2

%========================================================

\section{Customers}

TODO Chapter 2

%========================================================

\section{Proposed Location}

TODO Chapter 2

%========================================================

\section{Pricing and Positioning Strategy}

TODO Chapter 2

\subsection{Operational Costs}

TODO Chapter 2

\subsection{Competitive Analysis}

TODO Chapter 2

\subsection{Return on Investment}

TODO Chapter 2

%========================================================

\section{Sales and Distribution}

TODO Chapter 2

\subsection{Sales Strategy}

TODO Chapter 2

\subsection{Distribution Methods}

TODO Chapter 2

\subsection{Transaction Process}

TODO Chapter 2

%========================================================

\section{Advertising and Promotion}

TODO Chapter 2

\subsection{Advertising}

TODO Chapter 2

\subsection{Sales Promotion}

TODO Chapter 2

\subsection{Publicity}

TODO Chapter 2

\subsection{Web}

TODO Chapter 2
