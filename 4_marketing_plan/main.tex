{\let\cleardoublepage\relax \chapter*{Marketing Plan}}
\addcontentsline{toc}{chapter}{Marketing Plan}

% guidelines: https://www.thebalancesmb.com/writing-the-business-plan-section-5-2947030

%========================================================

\section{Industry Background}

\subsection{Market Size} \label{market.size}

According to QY Research, North America and Europe are currently holding the largest share for the privacy management software market \cite{qy.2020}. In 2019, this market size was estimated at a US \$808.8 million and estimated to reach a US \$6.1 billion by the end of 2026 with a compound annual growth rate of 33.1\% from 2021 through 2026. These estimates were based on privacy management software trends in North America, Europe, China, Southeast Asia, India, and Central and South America.

\subsection{Market Trends} \label{marketing.trends}

With the passing of the General Data Protection Regulation (GDPR) for European Union citizens and large data breaches exposing the personal information of hundreds of millions of consumers, privacy awareness has led to the passing of privacy regulations across an increasing amount of countries \cite{privacypolicies.2019}. Other countries, including Brazil, Canada, Australia, and various U.S. states have enacted, or began enforcing, their privacy laws with heftier fines or sanctions. The primary difference between regulations being passed now, in contrast to the past, are the heavy fines and sanctions regulators can impose on companies who are non-compliant \cite{tr.2020}. These sanctions and fines are designed to force businesses to comply with regulations and how they process their customers information.

In a survey conducted by Auxier, Rainie, Anderson, Perrin, Kumar, and Tuner, 62\% of Americans believe it is not possible to go through daily life without having their data collected, 81\% believe that they have very little to no control over their information companies collect about them, and 72\% feel that all, or almost all of what they do online is being tracked \cite{pewresearch.2019}. The survey also revealed that most Americans are not confident that companies would publicly admit to misusing their customers data, but still say data collection and usage is acceptable for processing in some ways.

Citizens of countries belonging to the European Union, 1 year since the passing of GDPR, are rapidly becoming aware of their privacy rights \cite{eucomm.2019}. A survey conducted by the European Union Commission against 27,000 Europeans revealed that 73\% of citizens have heard of at least one of their privacy rights with the highest awareness being the right to access their own data (65\%), the right to correct their information a company has on them (61\% ), and the right to opt-out of direct marketing (59\%).

Other trends involve the usage of smart mobile devices. According to the FTC, information from smart phones, such as location information, is considered sensitive \cite{ftc.mobile.2012}. The location information can be used to provide, unwanted, targeted marketing to consumers based on their movements. Mobile applications can use the hardware identifiers associated with devices to uniquely identify consumers or potentially access utilities no required for the intended purpose of the application \cite{tama.mobile.2012}.

\subsection{Growth Potential and Opportunity}

In the \nameref{market.size} section, QY Research identified a compound annual growth rate of 33.1\% from 2021 through 2026. The \nameref{marketing.trends} section identified emerging trends such as: privacy awareness, information control, and concerns with smart mobile devices. With a rapidly growing market and rising awareness and concerns with privacy, we can anticipate that governments will further regulate how companies can process consumer information. With further regulations, the demand for consent management vendors will grow (as predicted by QY Research) and  automated privacy auditing services to ensure privacy risk is mitigated by the vendors will be in similar demand. 

\subsection{Market Barriers}

Even with governmental regulations, hefty fines, and sanctions; adoption of consent management platforms (CMP) isn't widespread and integrations often don't meet compliance regulations \cite{nouwens.2020}. In an audit conduced by Aarhus, Cambridge, and UCL Universities, of the top 10,000 websites in the UK, only 11.8\% met the minimal requirements set by their audit based on European law. Of this sample, only 20.35\% of websites report to use a CMP. In addition to the top 10,000 websites, in a sample of 1,000 consent management platform vendors, 95.8\% provided either no consent choice or confirmation only.

%\noindent
A long-term study of the impact of GDPR on cookie placement also revealed that \cite{trevisan.2019}:

\begin{enumerate}
  \item 49\% of websites placed cookies before receiving consent.
  \item 28\% of websites didn't provide any consent mechanism.
  \item The percentage of websites violating GDPR stayed constant over the course of 4 years, indicating any consent mechanisms implemented were ineffective.
\end{enumerate}

%\indent

\noindent
The primary barrier for Cereus to successfully provide privacy auditing services will be industry acceptance. With many vendors offering consent management solutions, a majority of integrations failed to meet compliance standards \cite{nouwens.2020}. We will have to differentiate ourselves from consent management providers looking for a piece of a booming market and emphasize that we are not a consent management solution. Cereus is an auditing and risk mitigation solution that evaluates integrations with CMPs. Should a CMP integration fail to adhere to the defined rules within our auditing system, those failures will be reflected in the audit report.

\subsection{Market Changes}

As an automated privacy auditing company, Cereus is moderately affected by fluctuations in the privacy software market due to changes in privacy legislation and regulation. Response to changes will, generally, be the responsibility of our customers -- they will be able to alter their business rules within our systems to ensure that any new legislation is accounted for. To support our customers, our professional services staff will receive training on any new or changes to legislation.

In the event legislation expands or is implemented in geographic locations that we do not yet support, our Engineering staff will make the appropriate changes prior to the enforcement of the legislation.

%========================================================

\section{Products and Services}

Customers with a privacy program may see auditing as the next logical step towards ensuring compliance and privacy risk mitigation measures are being followed across all of their properties. Depending on size of their organization, manual auditing may be a possibility. Customers will also have to consider the costs associated with manual auditing, the possibility for human error, and the frequency in which the audits must be conducted. 

Other customers looking for privacy management solutions may not consider auditing and risk management as the first priority. They understand that their websites need to comply with relevant laws and a consent management platform can assist them with compliance. These customers will look through our products and services and will quickly identify that we do not offer a consent management solution, so why would they choose our services?

Cereus agrees that a consent management platform seems like the first logical step when viewing privacy legislation trends and actions your competitors may be taking. We also believe that a consent management platform may not be required given your audience, business practices, and tooling. After conducting an audit and identifying any privacy risks your businesses are susceptible to, you will be able to determine an appropriate course of action. That course may include implementing a consent management platform or possibly using open-source solutions that will mitigate risk.

With the highly customizable products and flexible pricing (even free) we offer, customers will see the potential to: identify whether or not they need a consent management platform, the associated risk of not implementing one, and the effectiveness of their integration should they implement a consent management platform.

%--------------------------------------------------------

\subsection{Auditor and Reporting}

Any customer, whether they have implemented a consent management platform or not, will first look towards how auditing and reporting will be provided. More importantly, what information the audit contains and how it can be distributed.

\subsubsection*{Unique Selling Proposition}

Define your compliance rules and actively monitor all your properties to ensure that they meet your compliance standards.

\subsubsection*{Features and Benefits}

\begin{itemize}

\item Customizable

  Define your compliance rules within our system for our proprietary crawler to validate against your websites.

\item Interactive

  Verify the results of your audit in our interactive report. Identify which of your properties failed to meet your compliance rules and where the infringement occurred on the website -- down to the line of code causing the infringement.

\item Exportable

  Export your audit into a distributable format to share with your stakeholders.

\item Automated

  Schedule audits to reflect your development cycle and quickly identify compliance infringements before they are distributed to your customers.

\end{itemize}

%--------------------------------------------------------

\subsection{Notifications and Alarms}

Customers may wonder how long an audit will take to run against their websites and, unfortunately, there is no definitive answer. Audits depend on how many pages the crawler must scan and the execution speed of your website. All of our customers, except those using the free tier, may specify who, if anyone, will receive notifications of when audits complete for each property, as well as an alarm should compliance checks fail.

\subsubsection*{Unique Selling Proposition}

Allow Cereus to grant you peace of mind by notifying you should one of your websites fall out of compliance.

\subsubsection*{Features and Benefits}

\begin{itemize}

\item Customizable

  Specify which stakeholders, at an organization or property scope, to notify when audits complete, and compliance checks fail for any of your websites.

\item Prompt

  No need to step away from your task to check on the status of your audits, allow Cereus to notify you as soon as your audit is complete.

\end{itemize}

%--------------------------------------------------------

\subsection{Classification and Recommendation Engine}

All our customers, excluding those on the free tier, will see our audit system provide a risk score, residual risk score, and suggest a category for partners running on your website that you have yet to review. Many will ask why such functionality is provided as their legal department will determine the rules associated with a partner. To assist with the evaluation of partners, we've aggregated resources so you don't have to.

\subsubsection*{Unique Selling Proposition}

Allow us to assist you with establishing your business rules by providing insights into the operations of your partners and suggest a course of action for your websites.

\subsubsection*{Features and Benefits}

\begin{itemize}

\item Automated

Cereus will automatically calculate the risk score and residual risk score should a partner load on your site in specific geographic locations. This score is based on the actions taken by our other customers, vendor information, and your privacy policy.

\item Productivity

Make informed decisions about mitigating any risk associated with your partners through the resources we've collected pertaining to their operations.

\end{itemize}

%--------------------------------------------------------

\subsection{API}

Organizations actively developing their websites will look for functionality that enables them to integrate privacy compliance and risk management into their daily operations. Our application programming interface (API) provides a variety of features to support a continuous integration system. This is available for our professional, enterprise, and select customers only.

\subsubsection*{Unique Selling Proposition}

Automate privacy compliance and risk management by integrating our API into your business operations.

\subsubsection*{Features and Benefits}

\begin{itemize}

\item Efficiency

Include privacy compliance and risk management in your continuous integration system. Run audits on-demand and let us send the results to your systems.

\end{itemize}

%--------------------------------------------------------

\subsection{Professional Services}

Not all organizations have the time or resources to dedicate towards running a privacy audit and risk management division. Professional and enterprise customers have the option to allow our professional services staff configure our systems to meet your privacy compliance and risk management requirements.

\subsubsection*{Unique Selling Proposition}

Allow us to assist your organization by configuring our systems to meet your privacy compliance and risk management requirements.

\subsubsection*{Features and Benefits}

\begin{itemize}

\item Support

Staff readily available to assist with your integrations, setup, and general questions.

\item Active Management

Our professional services staff can configure and manage our systems to meet your privacy compliance and risk management requirements.

\end{itemize}


%========================================================

\section{Customers}

We offer flexible subscriptions for any sized organization that's looking for privacy compliance auditing. It may, however, be difficult to market smaller companies that our services are worth mitigating the risk of being fined. Our customers will primarily be medium-to-large organizations, with multiple websites, who are currently, or will be, working with a consent management system.

\subsection{Demographic Considerations}

The geographic location of our customers and their client-base are some factors to consider when defining the demographics of our customers. The location of our customers and their client-base define the regulations they have to adhere to. Should an organization chooses to restrict their client-base due to data and privacy regulations, our services may not be required.

%========================================================

\section{Proposed Location}

Cereus will be incorporated in the United States. There will be no leased or physically owned real estate. Correspondence pertaining to Cereus will operate through a PO Box. 

%========================================================

\section{Pricing and Positioning Strategy} \label{price.position.strat}

The pricing of Cereus's products is determined by estimating the profit margin of each pricing tier, excluding professional services. We aim for a profit margin of 90\% (+/-5\%) during our first year. As our company grows, it'll be our goal keep these margins around 12\% as more software, support, and marketing staff are required. 

To estimate the profit margin for each tier, we monitored the actions of a full professional tier organization over a month during a beta phase. The monitoring revealed an average of about 2,500 API requests per user were made, with an average of 1.2GB of traffic, and the organization scanned 10 websites with 25 pages, twice. 20 notification and web hook events were sent, averaging 1kB in size. Each scan resulted in an average report size of 2.5kB. 

After the beta phase concluded, Cereus's founders created an isolated environment and simulated the actions of the professional organization. 24 hours after the simulation, Amazon's cost explorer identified a spike in projected costs of \$304.72 with an average of \$10.15 per day.

The cost explorer identified the following components as having the most significant impact in the projected costs. We derived an equation for each of the components, using Amazon's pricing and the features available for each of our subscription tiers (Table \ref{table.cereus.pricing}), to compute the estimated cost. Each equation was able to accurately compute the amount reflected in the cost explorer within a 4\% margin of error.

\begin{itemize}

\item Loadbalancer (API) Network Traffic

\( ELB_{est} = (0.008*(N_{users}*B_{api})) + (N_{users}*2500/1,000,000)*0.40) \)

\item Cloudfront Network Traffic

\( CF_{est} = (0.085 * N_{users} * 1.2_{gb}) \)

\item S3 Storage Costs

\(S3_{est} = (0.023 * (N_{websites} * N_{pages} * N_{scans} * 0.0000026_{gb}))\)

\item NAT Gateway Traffic

\(
  NAT_{est} = (0.045 * (N_{websites} * N_{pages} * N_{scans} * 0.0000026_{gb})) + \\
  (0.045 * (N_{websites} * N_{pages} * N_{scans} * 0.000001_{gb}))
\)

\end{itemize}

The costs between our queue, email, scan, database backup, and audit systems were also reflected in the cost explorer. Due to these systems being part of our standard infrastructure operations, as discussed in the \nameref{section.operational.costs} section, the costs associated with these had to be estimated.

A second test against the websites a beta tester belonging to the personal tier was conducted to help determine a constant value for the additional costs associated with our internal systems. Report sizes were smaller due to the tier being limited to a total of 25 pages, in contrast to 250 in the professional tier -- but only within a few hundred bytes due to the file compression we apply to our audits. Our estimated costs increased \$2.04 per day with this test.

Given the two tests, a constant value \(C\) equal to 19.992 (rounded to 20), multiplied by the sum of the associated cost of each component, estimated the monthly costs associated with both tests within a 2\% margin of error. Giving us the equation:

\[
  Plan_{cost} = C * (ELB_{est} + CF_{est} + S3_{est} + NAT_{est})
\]

%%

\subsection{Operational Costs} \label{section.operational.costs}

Operational costs are broken into two main categories: infrastructure and administration. Infrastructure includes all technology requirements to support operations. Administration includes expenses pertaining to wages, marketing, accounting, and professional services.

\subsubsection{Infrastructure Costs} \label{infra.costs}

Cereus will manage its infrastructure through Amazon Web Services (AWS). AWS offers an alternative solution to physical infrastructure management at a competitive rate. Cereus will require the following services from Amazon to support the auditing of 100 websites:

\begin{enumerate}[\indent {}]

\item \textbf{1 A1.Medium Spot Instance}

The scan servers to collect information from our customer's websites can be reserved spot instances that operate only on demand. These servers come at a discounted rate, but introduce some latency between a customer's audit request and its completion. Each server is expected to be able to scan up to 10 websites at a time.

Pricing can be calculated with the following formula where \( t \) is time in hours and \( n_{a1s} \) is the number of servers:

\[ Scan_{cost} = 0.0049tn_{a1s} \]

\item \textbf{2 A1.Medium Reserved Instances}

Cereus operates on a client-server model, in which the client will operate as "serverless" through S3 and a CDN. The API services for server portion must be available at all times with limited latency. This comes at an additional cost in contrast to spot instances, but will ensure our customers can request information on-demand. 

Pricing can be calculated with the following formula where \( t \) is time in hours and \( n_{a1r} \) is the number of servers:

\[ Api_{cost} = 0.0255tn_{a1r} \]

\item \textbf{2 S3 buckets}

All audit information will be stored in a private Amazon S3 bucket. Cereus's landing page and user interface will be hosted through S3 as well.

Pricing can be calculated with the following formula where \( g_{s3} \) is the size of all S3 buckets in gigabytes (for the first 50 terabytes):

\[ S3_{cost} = 0.023g_{s3} \]

\item \textbf{1 Cloudfront Instance Across All Edge Locations}

Cloudfront will serve as our CDN layer for our landing page and user interface. This will significantly reduce the latency of serving content to our customers.

Pricing can be calculated with the following formula where \( g_{cf} \) is the size of all traffic in gigabytes (for the first 10 terabytes):

\[ CF_{cost} = 0.085g_{cf} \]

\item \textbf{1 Network Address Translation Gateway (NAT)}

To protect our proprietary technologies and customer's data, Cereus's services will primarily reside in a private network within AWS. The NAT will allow our internal network to communicate with the internet. 

Pricing can be calculated with the following formula where \( g_{nat} \) is the size of all traffic in gigabytes and \( t \) is the total time in hours:

\[ NAT_{cost} = 0.045t + 0.045g_{nat} \]

\item \textbf{1 Replicated Postgres Instance}

Cereus will store its customer information and configurations in a postgres database. Data will be replicated across two instances for high availability.

Pricing can be calculated with the following formula where \( t \) is the total time in hours:

\[ RDS_{cost} = 0.072t \]

\item \textbf{1 T2.Small Elasticache Instance}

Scheduled scans and notifications will operate through a queue-consumer implementation. Amazon's Elasticache service will serve as the queue.

Pricing can be calculated with the following formula where \( t \) is the total time in hours:

\[ ELC_{cost} = 0.034t \]

\item \textbf{API Loadbalancer}

Though only 1 API instance is needed for minimum operations, should that 1 API server fail, Cereus's customers will not be able to access their configurations or audits. Running 2 API instances with a Loadbalancer will prevent all operations from halting due to one server going offline.

Pricing can be calculated with the following formula where \( g_{elb} \) is the size of all traffic in gigabytes and \( t \) is the total time in hours:

\[ ELB_{cost} = 0.025t + 0.008g_{elb} \]

\item \textbf{1 Route53 Hosted Zone}

Route53 is Amazon's domain-name server resolver. Cereus's domain service provider can route requests from our domain to our servers, through Route53, hosted on Amazon.

Pricing can be calculated with the following formula where \( r_{m} \) is the number of requests (per million, up to 1 billion):

\[ R53_{cost} = 0.50 + 0.40r_{m} \]

\end{enumerate}



\noindent
Using the cost formulas defined for each Amazon service, we can calculate our operations costs with the following formula (simplified):

\[
  C = 0.176t + 0.0049tn_{a1s} + 0.0255tn_{a1r} + 0.023g_{s3} + 0.085g_{cf} + 0.045g_{nat} + 0.008g_{elb} + 0.40r_{m} + 0.50
\]

\noindent
Preliminary testing estimated that the following variables would accomodate the auditing of 100 websites on the professional subscription tier:

\begin{itemize}

\item 2 Scan Servers
\item 1 Reserved API Instance
\item 1.5GB S3 storage/mo
\item 5GB of traffic through Cloudfront
\item 1GB of traffic through the NAT
\item 10 million requests

\end{itemize}

\noindent
With Amazon's 720 hour billing cycle (30 days) \cite{aws.calc.2020} and the operations cost formula, the estimated monthly cost for infrastructure operations would be around \$157.14.

\subsubsection{Administration Costs} \label{section.admin.costs}

\begin{itemize}

\item Wages and Benefits

The only employees of Cereus will be its three founders, who will not take a salary, dividends, or benefits during its first year of operations. Two of the founders will contribute to the company part-time while they contribute to their main source of income, the other will dedicate a full-time schedule towards the company's operations. The refusal of monetary compensation during the first year will allow the founders to invest any gains back into the company and help it grow.

\item Marketing

Cereus's founders will conduct initial marketing campaigns through professional social media platforms, popular search engines, and our website. These, initially, come at little to no cost. A budget of \$250 per month to serve targeted advertisements on popular search engines and social media platforms will be allocated. After assessing the performance of our marketing efforts, we will re-evaluate our strategies to grow our customer base.

\item Accounting

According to SCORE, in 2015, most small business owners spent at least \$1,000 in accounting administration costs, internal expenses, and legal fees each year \cite{score.2016}. 31\% of business owners reported to spending between \$1,000 and \$5,000. Taking into consideration the subscription model Cereus follows (Table \ref{table.cereus.pricing}), \$200 per month will be reserved for accounting purposes.

\item Professional Services

Cereus's professional services and support will be ran by the three founders with one dedicated 40 hours per week. In the event the demand for professional services exceeds the capabilities of the founders, Cereus will hire part-time support staff as needed.

\end{itemize}

\subsection{Competitive Analysis} \label{competitive.analysis}

Cereus provides automated privacy compliance auditing and risk management services. This is an extremely niche and relatively new market that is often conglomerated with the information security auditing industry. Most of Cereus's primary competitors will be those who conduct audits manually. Secondary competitors will be consent management platforms that have the capabilities to implement automated auditing into their systems. Table \ref{table.cereus.overview} provides a competitive overview of our services, whereas Table \ref{table.cereus.companal} evaluates our competitors.

\begin{table}[H]
  \caption{Cereus competitive overview.}
  \centering
  \setlength\tabcolsep{5pt}
  \def\arraystretch{1.2}%

  \begin{tabularx}{\textwidth}{p{2cm}|p{6.2cm} p{1.5cm} p{1.75cm} p{1.85cm} }

    \hline

    Factor & Cereus & Strength & Weakness & Importance  \\
  
    \hline
  
    Products & Automated and configurable. & \centering{x} & & High \\

    Price & Fixed and flexible pricing models. & \centering{x} & & Medium \\

    Quality & High. & \centering{x} & & High \\

    Selection & Limited to system-defined geographic locations. & & \centering{x} & High \\

    Service & Support limited to professional and enterprise customers. & & \centering{x} & Medium \\

    Reliability & Cloud service replicated across multiple regions. & \centering{x} & & High \\

    Stability & New company. & & \centering{x} & High \\

    Expertise & Founders are privacy, software, product, and marketing experts. & \centering{x} & & Medium \\

    Reputation & Unknown. & & \centering{x} & Medium \\

    Location & United States. & & & Low \\

    Appearance & Audits and configuration will be conducted through our website. & & & Medium \\

    Sales & Primarily B2B with some B2C sales. & & & Low \\

    Credit Policy & No credit. Monthly or yearly subscriptions. & & & Low \\

    Advertising & Digital/Online. & & & Low \\

    Image & Opinion. Unestablished. & & \centering{x} & Medium \\

    \hline

  \end{tabularx}
  \label{table.cereus.overview}
\end{table}

%%%%%%%%%%%%%%%%%%%%%%

\begin{sidewaystable}
  \caption{Cereus Competitive Analysis.}
  \centering
  \setlength\tabcolsep{5pt}
  \def\arraystretch{1.2}%

  \begin{tabularx}{\textwidth}{p{2.5cm} | p{6.5cm} | p{6.5cm} | p{6.5cm}}

    \hline

    Factor & Altius & OneTrust & Crownpeak  \\

    \hline
  
    Products & Manual IT security, privacy, and risk consultation.  & CMP, data mapping, and professional services. & CMS, CMP, and accessibility services. \\

    \hline

    Price & Fixed contract rate. & Fixed and customized rates. & Fix subscription rates. \\

    \hline

    Quality & High: customer reviews. & High: customer and business reviews. & Medium: customer and business reviews. \\

    \hline

    Selection & Broad range of security, risk, and compliance consulting services. & Configurable, automated, CMP solutions. & Limited to an automated CMP solution (pertaining to privacy). \\

    \hline

    Service & Consulting. & SaaS and consulting. & SaaS and consulting. \\

    \hline

    Reliability & Unknown. & High. & High. \\

    \hline

    Stability & Well established (1993). & Established (2016). & Well established (2001). \\

    \hline

    Expertise & 25 years of audit and security experience. & Consent management, privacy regulation, and data governance. & Digital experience management. \\

    \hline

    Reputation & Thorough. & CMP leader. & Digital experience leader. \\

    \hline

    Location & United States. & United States. & United States. \\

    \hline

    Appearance & Antiquated web presence. & Professional, modern, web presence. & Professional, modern, web presence. \\

    \hline

    Sales & Consulting. & B2B, B2C, and consulting. & B2B, B2C, and consulting. \\

    \hline

    Credit Policy & Unknown. & None. & None. \\

    \hline

    Advertising & Web presence. & Web, social media, and search engine advertising presence. & Web, social media, and search engine advertising presence. \\

    \hline

    Image & Professional consultancy. & Professional data and privacy SaaS provider. & Professional DMX SaaS provider. \\


\end{tabularx}
\label{table.cereus.companal}
\end{sidewaystable}

\subsection{Return on Investment}

Table \ref{table.cereus.breakeven} highlights each of Cereus's subscription tiers. Each column reflects the tier's pricing, the estimated profit per month (using the equation described in the \nameref{price.position.strat} section), the estimated profit margin (without professional services), and the number of subscriptions required to break-even given the infrastructure costs (\$157.14), accounting budget (\$200.00), and marketing budget (\$250.00).

\begin{table}[H]
  \caption{Cereus subscription break-even analysis.}
  \centering
  \begin{tabularx}{\textwidth}{|X|X|X|X|X|X|}
    Tier & Price & Cost & Profit & Margin & Break-Even \\
  
    \hline
  
    Free & \$0.00 & \$0.10 & -\$0.10 & 0 & N/A \\

    \hline
  
    Personal & \$39.00 & \$2.06 & \$36.94 & 0.947 & 17 \\ 

    \hline
  
    Professional & \$99.00 & \$10.32 & \$88.68 & 0.896 & 7 \\ 

    \hline

    Enterprise & \$1,299.00 & \$55.86 & \$1,243.00 & 0.957 & 1 \\ 

  \end{tabularx}
  \label{table.cereus.breakeven}
\end{table}

%========================================================

\section{Sales and Distribution}

\subsection{Sales Strategy}

Initially, all purchases of our services will be available only through our website. The founders will be available via email to handle any purchase inquiries. In the event an organization inquires about our Enterprise or Select tier subscriptions, the founders will conduct a product demonstration if requested.

\subsection{Distribution Methods}

Cereus offers software-as-a-service with professional services available for professional and enterprise customers. There are no physical products to distribute and any purchases will be immediately available to our customers.

\subsection{Transaction Process} \label{transaction.process}

Transactions will be handled through Cereus's website. After a customer registers their organization, they will be prompted to enter in a credit card. Once entered in, a billing profile will be set up with our credit processing provider, Authorize.net. Accounts will be billed the first of every month. In the event that a transaction fails to process, Cereus will no longer authorize the auditing of any websites under the organization and the customer will be notified. If an account is delinquent for more than one billing cycle, the account will be closed.

\subsubsection{Returns and Refunds}

For monthly subscriptions, Cereus will not offer returns for the purchase of our services. Should an organization with a yearly subscription inquire about canceling their subscription, Cereus will charge a 10\% processing fee against the cost of the subscription.

%========================================================

%%%

\section{Advertising and Promotion} \label{marketing.ad.promo}

\subsection{Advertising and Publicity}

As outlined in the \nameref{section.admin.costs} section, Cereus will focus its digital advertising efforts towards professional social network platforms and search engines. We believe that pure digital approach will allow us to gain traction in the privacy auditing industry. This may lead to technical publishers and journals reviewing our products and services -- offering more exposure while keeping our marketing costs low.

We encourage our customers to review our products for further exposure, but there has been mixed findings regarding the impact of review valence and customer responses, specifically to the possibility of reviews being moderated \cite{maslowska.reviews.2017}. Customers aware of deceptive practices with reviews were found to have different intentions, and attitudes, towards products based on the review valence \cite{karabas.reviews.2020}. Should our products ever get perfect reviews across services that support the rating of our company, we anticipate these customers would approach us with skepticism. Poor reviews, deceptive or not, may prevent customers from even considering us.

To help mitigate negative reviews, initially, we will rely heavily on customers testing our products and services with our free tier before purchasing a subscription. Our monthly subscriptions are non-refundable, which we suspect will be a major factor for negative reviews. Initially, we will routinely monitor reviews across popular platforms such as Google Reviews. As our company grows, the marketing staff will be able to actively monitor reviews to address customer concerns and complaints.

\subsection{Sales Promotion}

Cereus will offer a 5\% early adopters discount to all customers who purchase a subscription within the first year of our founding.

\subsection{Web}

Our websites landing page will feature all of our products and services. It will support all major browsers and be responsive to mobile devices. We will present a call-to-action for all visitors, prompting them to try our services for free with our free tier. We will advertise that, after evaluating the products available in the free tier, the customer can upgrade to any one of our subscriptions at any time.
 

\section{Sales Forecast} \label{section.sales.forecast}

Based on the marketing and promotional strategies outlined in the \nameref{marketing.ad.promo} section, a 12-month sales forecast was generated under the following assumptions \cite{sales.forecast.2020}:

%\begin{itemize}

\subsubsection*{20,000 LinkedIn impressions}

With a budget of \$130.00 for LinkedIn advertising, at the cost of \$6.59 per 1,000 impressions and a click rate of 0.39\%, we can expect up to 78 potential customers to visit our website through LinkedIn per month \cite{wilcox.2020}.

\subsubsection*{42,000 Google Ad Words impressions}

Google Ad Words has an average click-through rate of 2\% at a cost of \$2.80 per 1,000 impressions and clicks. We can expect 840 potential customers to visit our website, per month, with a budget of \$118.00 advertising with Google \cite{adstag.2020}.

\subsubsection*{1\% Conversion rate}

A 1\% conversion rate is defined as a customer visiting our website from one of our advertising partners and creating a free-tier account.

\subsubsection*{85\% of all accounts are free subscriptions}

85\% of accounts within Cereus's systems are anticipated to be of the free tier. Potential customers who engage with our advertisements and calls to action will create a free account before subscribing.

\subsubsection*{10\% of all accounts are personal subscriptions}

After customers evaluate our free tier, we estimate that 10\% of all accounts in our systems will be the personal tier.

\subsubsection*{3\% of all accounts are professional subscriptions}

Professional subscriptions are expected to require product demonstrations. Companies that view our advertisements will visit our website, potentially create a free account, and reach out to our professional services staff to view our more advanced features. 3\% of accounts are expected to be professional subscriptions. 

\subsubsection*{Less than 2\% of all accounts are enterprise and select subscriptions}

Enterprise and Select subscriptions will require more than a free account evaluation to convince a customer to purchase our services. These customers will want a thorough walkthrough of our products and see how we can assist their organization. Less than 2\% of all our accounts will be of the enterprise and select tiers.

%%%%%%%%%%%%%%%%%%%

\begin{sidewaystable}
\begin{table}[H]
  \caption{Cereus sales forecast.}
  \centering
  \setlength\tabcolsep{5pt}
  \def\arraystretch{1.2}%

  \begin{tabularx}{\textwidth}{X X X X X X X X X X X X X}

    \multicolumn{13}{c}{\textbf{Monthly Subscriptions}} \\

    \hline
    
    \multicolumn{1}{c|}{\textbf{Tier}} & \textbf{1} & \textbf{2} & \textbf{3} & \textbf{4} & \textbf{5} & \textbf{6} & \textbf{7} & \textbf{8} & \textbf{9} & \textbf{10} & \textbf{11} & \textbf{12}  \\

    \hline

    \multicolumn{1}{c|}{Free} & 6 & 5 & 8 & 8 & 6 & 6 & 8 & 8 & 7 & 6 & 5 & 5  \\

    \multicolumn{1}{c|}{Personal} & 2 & 2 & 0 & 0 & 0 & 0 & 0 & 0 & 1 & 2 & 3 & 2  \\

    \multicolumn{1}{c|}{Professional} & 0 & 0 & 0 & 0 & 0 & 0 & 0 & 0 & 0 & 0 & 0 & 0  \\

    \multicolumn{1}{c|}{Enterprise} & 0 & 0 & 0 & 0 & 0 & 1 & 0 & 0 & 0 & 0 & 0 & 0  \\

    \hline

    \multicolumn{1}{c|}{\textbf{Customers}} & 8 & 15 & 23 & 31 & 36 & 43 & 51 & 59 & 67 & 75 & 83 & 90  \\

    \hline

    \multicolumn{1}{c|}{\textbf{Revenue}} & \$73.28 & \$146.66 & \$145.86 & \$145.06 & \$144.46 & \$1386.86 & \$1386.06  & \$1385.26 & \$1421.50 & \$1494.78 & \$1605.10 & \$1678.48  \\

  \end{tabularx}
  \label{table.sales.forecast}
\end{table}
\end{sidewaystable}
