{\let\cleardoublepage\relax \chapter*{Next Steps}}
\addcontentsline{toc}{chapter}{Next Steps}

Our business plan is best treated as a living document. As time progresses, the industry changes, and our company grows; our business plan will adapt. Aspects of our company, such as our infrastructure costs and marketing strategy, are expected to change frequently. Other components, such as our mission and objectives, will remain relatively the same. Products and services will always respond to customer feedback and legislation changes, but the demand for these changes is currently unknown.

\section{Situation}

All of our expenses, revenues, and projections are informed estimates. Based on these estimates and projections, it is reasonable to believe that our company will be profitable within the first year -- so long as marketing operations are successful. These are, however, only estimates and may not represent our observations over time. Should there be a significant difference between our projections and observations, we will refine our business plan to achieve our goal of being profitable within the first year. 

\section{Target}

Our primary target is to be profitable within the first year. Other key metrics include: the number of paid subscriptions, number of audits conducted each month, the number of websites being actively monitored within our system, and the total number of website pages we audit. Ideally we would have at least 17 personal subscriptions auditing 85 pages once a month -- this will, at least, allow us to break even the first year.

\section{Proposal}

Our path to achieving our goals will rely heavily on our marketing strategy and demonstrations. The more customers we acquire, the more likely we'll be able to support and improve our ventures. Though we have well informed projections on our operational costs and sales, these may not be accurate. Should this be the case, we will have to alter our strategies and potentially seek outside investment.  